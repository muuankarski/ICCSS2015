%\documentclass[draft, 10pt,fleqn,twoside]{article}
\documentclass{article}\usepackage[]{graphicx}\usepackage[]{color}
%% maxwidth is the original width if it is less than linewidth
%% otherwise use linewidth (to make sure the graphics do not exceed the margin)
\makeatletter
\def\maxwidth{ %
  \ifdim\Gin@nat@width>\linewidth
    \linewidth
  \else
    \Gin@nat@width
  \fi
}
\makeatother

\definecolor{fgcolor}{rgb}{0.345, 0.345, 0.345}
\newcommand{\hlnum}[1]{\textcolor[rgb]{0.686,0.059,0.569}{#1}}%
\newcommand{\hlstr}[1]{\textcolor[rgb]{0.192,0.494,0.8}{#1}}%
\newcommand{\hlcom}[1]{\textcolor[rgb]{0.678,0.584,0.686}{\textit{#1}}}%
\newcommand{\hlopt}[1]{\textcolor[rgb]{0,0,0}{#1}}%
\newcommand{\hlstd}[1]{\textcolor[rgb]{0.345,0.345,0.345}{#1}}%
\newcommand{\hlkwa}[1]{\textcolor[rgb]{0.161,0.373,0.58}{\textbf{#1}}}%
\newcommand{\hlkwb}[1]{\textcolor[rgb]{0.69,0.353,0.396}{#1}}%
\newcommand{\hlkwc}[1]{\textcolor[rgb]{0.333,0.667,0.333}{#1}}%
\newcommand{\hlkwd}[1]{\textcolor[rgb]{0.737,0.353,0.396}{\textbf{#1}}}%

\usepackage{framed}
\makeatletter
\newenvironment{kframe}{%
 \def\at@end@of@kframe{}%
 \ifinner\ifhmode%
  \def\at@end@of@kframe{\end{minipage}}%
  \begin{minipage}{\columnwidth}%
 \fi\fi%
 \def\FrameCommand##1{\hskip\@totalleftmargin \hskip-\fboxsep
 \colorbox{shadecolor}{##1}\hskip-\fboxsep
     % There is no \\@totalrightmargin, so:
     \hskip-\linewidth \hskip-\@totalleftmargin \hskip\columnwidth}%
 \MakeFramed {\advance\hsize-\width
   \@totalleftmargin\z@ \linewidth\hsize
   \@setminipage}}%
 {\par\unskip\endMakeFramed%
 \at@end@of@kframe}
\makeatother

\definecolor{shadecolor}{rgb}{.97, .97, .97}
\definecolor{messagecolor}{rgb}{0, 0, 0}
\definecolor{warningcolor}{rgb}{1, 0, 1}
\definecolor{errorcolor}{rgb}{1, 0, 0}
\newenvironment{knitrout}{}{} % an empty environment to be redefined in TeX

\usepackage{alltt}
\usepackage[utf8]{inputenc}
\usepackage[sc]{mathpazo}
\usepackage[T1]{fontenc}
%\usepackage{hyperref}
\usepackage{geometry}
\geometry{verbose,tmargin=1.0cm,bmargin=1.5cm,lmargin=1.5cm,rmargin=1.5cm}
\setcounter{secnumdepth}{2}
\setcounter{tocdepth}{2}
\usepackage{url}
\usepackage[unicode=true,pdfusetitle,
 bookmarks=true,bookmarksnumbered=true,bookmarksopen=true,bookmarksopenlevel=2,
 breaklinks=false,pdfborder={0 0 1},backref=false,colorlinks=false]
 {hyperref}
\hypersetup{
 pdfstartview={XYZ null null 1}}
\usepackage{authblk}

\title{Computational ecosystems for social science\thanks{Extended abstract submitted for International Conference on Computational Social Science in June 8-11, 2015 in Helsinki, Finland}}
\author[1]{Markus Kainu \thanks{markuskainu@gmail.com}}
\author[2]{Joona Lehtomäki \thanks{joona.lehtomaki@helsinki.fi}}
\author[3]{Juuso Parkkinen \thanks{juuso.parkkinen@iki.fi}}
\author[4]{Juha Yrjölä \thanks{juha.yrjola@iki.fi}}
\author[5]{Måns Magnusson \thanks{mans.magnusson@gmail.com}}
\author[6]{Leo Lahti \thanks{leo.lahti@iki.fi}}

\affil[1]{Aleksanteri Institute, University of Helsinki, Finland}
\affil[2]{Department of Biosciences, University of Helsinki, Finland}
\affil[3]{Reaktor Innovations Oy, Finland}
\affil[4]{Kansan Muisti ry, Finland}
\affil[5]{Linköping University, Sweden}
\affil[6]{Department of Veterinary Bioscience, University of Helsinki, Finland}
\IfFileExists{upquote.sty}{\usepackage{upquote}}{}
\begin{document}
%\SweaveOpts{concordance=TRUE}
  \maketitle
{\bf Keywords:} social science; elections; open government data; statistical programming; machine learning

\vspace{1mm}

\section{Title \& authors}

[TODO lisätäänkö Mikko Tolonen tekijälistalle? Lisäksi pitää kysyä
tahtooko Juha olla nyt mukana kun ei olekaan vaalijuttua +
kansanmuistia posterissa]


\section{Background}

The recent explosion in open data availability has created novel
opportunities for research, journalism and citizen
science. High-quality machine readable data streams are increasingly
available on political decision making, historical processes, welfare,
traffic, and other aspects of society. There is a great need for
analytical tools to take advantage of these new data streams in
computational social science, digital humanities, and related fields.


\section{Open source data analytics}

Efficient data analysis relies on customized, reproducible analysis
workflows that are best developed jointly by the user community.
Availability of ready-made algorithms for standard data analysis tasks
allows an individual researcher to avoid reinventing the wheel,
leaving more time to solve the specific research problems. Solutions
have emerged in data intensive research fields, such as bioinformatics
and particle physics, based on open source statistical programming
languages. In computational social sciences and digital humanities,
analogous statistical software libraries are now emerging and have a
huge potential to contribute to transforming the field. However, these
resources are currently highly scattered and come in various formats,
hindering wider adoption. Specific web-based tools are available, but
more flexible computational tools are urgently needed for fully
powered data processing and analysis.

\section{rOpenGov developer community and ecosystem}

rOpenGov is a statistical ecosystem focused on open source data
analysis algorithms relevant to computational social sciences and
digital humanities. We provide a discussion forum and flexible
algorithms for reproducible data analysis in these fields. We are a
community of independent package developers from various countries,
and we build on experiences learned from similar initiatives in other
fields, such as Bioconductor and rOpenSci.

rOpenGov is based on the R statistical programming language, which has
a versatile computational ecosystem with rich statistical modeling and
state-of-the-art visualization capabilities. We are actively
monitoring developments in other languages, such as Python and
Julia. The wide scope of the R language is essential for addressing
the diversity of analysis tasks. We complement the prevailing R
ecosystem with custom tools for computational social sciences and
digital humanities. The packages are distributed through Github
(ropengov.github.io). The project is maintained by a core team and a
number of independent contributors (see the rOpenGov site for the
up-to-date author list) [EI VARMAAN KANDE KAIKKIA AUTHOREITA LISTATA
POSTERISSA VAIKKA PERIAATTEESSA VOISI - ENSINNÄKIN LISTA PÄIVITTYY
JATKUVASTI JA JOS OIKEIN HUONO TUURI KÄY NIIN JOKU VOI VAHINGOSSA
UNOHTUA JA LOUKKAANTUA].


\section{Example: Eurostat tools}




\section{Social coding}

The ecosystem enables rapid development of scalable and interoperable
software and provides tools to expand the quantitative methods
base. The advantages of the open development model include:

\begin{itemize}

\item Open source: We use GitHub for shared version control. All
  contributions are openly licensed. This guarantees that the tools
  are freely available and the international scientific community
  remains the owner of the research software.

\item Reproducible documentation: High-quality documentation is critical
  for package usability. We provide online tutorials with fully
  reproducible documentation on how to access and analyse specific
  data sources, and to report the statistical results.

\item Transparent research: The programmatic approach makes it possible to
  publish the data analysis steps from raw data to the final results
  in full detail. To exemplify this, we publish reproducible case
  studies based on open data and algorithms in the rOpenGov blog.

\item Standardization: A community-driven approach helps to pool scarce
  research resources and develop common standards for data
  analysis. Joint development ensures that the applicability of the
  tools extends beyond individual data sets and is compatible with
  other tools. Whereas different research projects can utilize the
  same standard algorithms to access and preprocess the data, the
  source code can be flexibly adapted to different tasks.

\end{itemize}


\section{Further resources}

The rOpenGov tools are distributed as R packages, including for
instance:

\begin{itemize}
 \item bibliographica: Bibliographic data analysis
 \item estc: British Library English Short Title Catalogue analytics
 \item eurostat: Eurostat open data analysis
 \item fennica: Finnish national bibliography analytics
 \item finpar: Finnish parliamentary data
 \item gisfin: Finnish geographic location information
 \item helsinki: Helsinki open data tools
 \item pxweb: R interface to PX-Web data (Statistics Finland \& Sweden etc.)
 \item rustfare: Russian open data
 \item sorvi: Finnish open government data
\end{itemize}

[TODO: NÄITÄ VOISI LISTATA LISÄÄKIN VEPPISIVULTA JA IHAN RYHMITELLÄ
JOHONKIN BOKSEIHIN?]


\section{Contact \& Contribute}

\begin{itemize}
 \item Web \& Blog: ropengov.github.io
 \item IRC: ropengov@Freenode
\end{itemize}

\section{References [ONKOHAN TARPEEN POSTERISSA? JÄTTÄISIN EHKÄ POIS]}

[1] S. Kasberger (2012). Grazwahl: Data Analysis and Visualizations of the communal elections in Graz.R package
[2] S. Fortunato and C. Castellano (2012). Physics peeks into the ballot box. Physics Today 65:74
[3] G. King, J. Pan and M. E. Roberts (2013). How Censorship in China Allows Government Criticism but Silences Collective Expression. American
Political Science Review, 107(02), 326–343
[4] M. L. Jockers (2013). Macroanalysis: Digital Methods and Literary History. University of Illinois Press.
[5] S. Chou, W. Li and R. Sridharan, Democratizing Data Science.
[6] D. Lazer, et al. (2009). Computational Social Science 323, 721–723

\end{document}
